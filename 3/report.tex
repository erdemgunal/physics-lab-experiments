\documentclass[12pt,a4paper]{article}
\usepackage[utf8]{inputenc}
\usepackage[turkish]{babel}
\usepackage{amsmath}
\usepackage{amsfonts}
\usepackage{amssymb}
\usepackage{graphicx}
\usepackage{float}
\usepackage{geometry}
\geometry{margin=2.5cm}

\title{DENEY 3: İNCE KENARLı MERCEK}
\author{Ad Soyad \\ Öğrenci No: \\ Grup No: }
\date{\today}

\begin{document}

\maketitle

\section{Deneyin Amacı}
İnce kenarlı merceğin odak uzaklığının bulunması.

\section{Teori}
Mercek denklemi:
\begin{equation}
\frac{1}{f} = \frac{1}{d_o} + \frac{1}{d_i}
\end{equation}

Büyütme oranı:
\begin{equation}
m = \frac{d_i}{d_o} = \frac{h_i}{h_o}
\end{equation}

\section{Deney Düzeneği}
\begin{itemize}
    \item Işık kaynağı
    \item İnce kenarlı mercek
    \item Ekran
    \item Cetvel
    \item Kağıt şerit
\end{itemize}

\section{Ölçümler ve Sonuçlar}

\subsection{Ölçüm Tablosu}
\begin{table}[H]
\centering
\small
\begin{tabular}{|c|c|c|c|c|c|c|}
\hline
n & $d_o$ (cm) & $d_i$ (cm) & $1/d_o$ (1/cm) & $1/d_i$ (1/cm) & $m = d_i/d_o$ & $f$ (cm) \\
\hline
1 & & & & & & \\
2 & & & & & & \\
3 & & & & & & \\
4 & & & & & & \\
5 & & & & & & \\
6 & & & & & & \\
7 & & & & & & \\
8 & & & & & & \\
9 & & & & & & \\
10 & & & & & & \\
\hline
\end{tabular}
\caption{Deneysel ölçümler}
\end{table}

\subsection{Grafikler}
\begin{figure}[H]
\centering
\includegraphics[width=0.8\textwidth]{graphs/reciprocal_distances.png}
\caption{$1/d_o$ - $1/d_i$ grafiği}
\end{figure}

\begin{figure}[H]
\centering
\includegraphics[width=0.8\textwidth]{graphs/focal_length_analysis.png}
\caption{Odak uzaklığı analizi}
\end{figure}

\section{Hesaplamalar}
Ortalama odak uzaklığı: $f_{ort} = $ \ldots cm

Standart sapma: $\sigma = $ \ldots cm

\section{Hata Analizi}
% Maksimum bağıl hata ve mutlak hata hesaplamaları

\section{Sonuçlar ve Değerlendirme}
% Teorik ve deneysel değerlerin karşılaştırılması

\end{document}
