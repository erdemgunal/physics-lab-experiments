\documentclass[12pt,a4paper]{article}
\usepackage[utf8]{inputenc}
\usepackage[turkish]{babel}
\usepackage{amsmath}
\usepackage{amsfonts}
\usepackage{amssymb}
\usepackage{graphicx}
\usepackage{float}
\usepackage{geometry}
\geometry{margin=2.5cm}

\title{DENEY 1: METALLERİN ÖZGÜl ISISININ TAYİNİ}
\author{Ad Soyad \\ Öğrenci No: \\ Grup No: }
\date{\today}

\begin{document}

\maketitle

\section{Deneyin Amacı}
Metallerin özgül ısısının ölçülmesi.

\section{Teori}
Bir maddeye herhangi bir iş yapılmaksızın enerji verilirse, maddenin sıcaklığı genellikle artar. 

Özgül ısı formülü:
\begin{equation}
Q = mc\Delta T
\end{equation}

Kalorimetri denklemi:
\begin{equation}
m_{Fe} \cdot c_{Fe} \cdot (100 - T_2) = (m_{su} \cdot c_{su} + m_{Cu} \cdot c_{Cu}) \cdot (T_2 - T_1)
\end{equation}

\section{Deney Düzeneği}
\begin{itemize}
    \item Demir parçası
    \item Kalorimetre
    \item Termometre
    \item Karıştırıcı
    \item Terazi
\end{itemize}

\section{Ölçümler ve Sonuçlar}

\subsection{Ölçüm Tablosu}
\begin{table}[H]
\centering
\begin{tabular}{|c|c|}
\hline
Ölçülen Büyüklük & Değer \\
\hline
Demir kütlesi ($m_{Fe}$) & \ldots kg \\
Kalorimetre kabı kütlesi ($m_{Cu}$) & \ldots kg \\
Su kütlesi ($m_{su}$) & \ldots kg \\
Başlangıç sıcaklığı ($T_1$) & \ldots °C \\
Son sıcaklık ($T_2$) & \ldots °C \\
\hline
\end{tabular}
\caption{Deneysel ölçümler}
\end{table}

\subsection{Hesaplamalar}
% Buraya Python'dan üretilen sonuçlar eklenecek

\subsection{Grafikler}
% Buraya Python'dan üretilen grafikler eklenecek
\begin{figure}[H]
\centering
\includegraphics[width=0.8\textwidth]{graphs/temperature_vs_time.png}
\caption{Sıcaklık-Zaman grafiği}
\end{figure}

\section{Hata Analizi}
% Hata hesaplamaları buraya eklenecek

\section{Sonuçlar ve Değerlendirme}
% Sonuçlar ve yorumlar buraya yazılacak

\end{document}
